\documentclass{article}

%%% Packages %%%

\usepackage[margin=1in]{geometry}
\usepackage{exercise}
\renewcommand{\ExerciseHeader}{%
  \par\noindent
  \textbf{\large \ExerciseName{} \, \ExerciseHeaderNB\ExerciseHeaderTitle\ExerciseHeaderOrigin}%
  \par\nopagebreak\medskip
}
\usepackage[utf8]{inputenc}
\usepackage[T1]{fontenc}
\usepackage{amsmath, amsfonts, amssymb, stmaryrd, bbold, url, hyperref}
\usepackage[version=4]{mhchem}
\usepackage{graphicx}

%%% Commands %%%

\newcommand{\proba}[1]{\mathbb{P}[#1]}
\newcommand{\var}[1]{\operatorname{Var}[#1]}
\newcommand{\cov}[1]{\operatorname{Cov}[#1]}
\newcommand{\esperance}[1]{\mathbb{E}[#1]}
\newcommand{\limitinftyn}{\xrightarrow[n \to{} \infty]{}}
\newcommand{\indicator}[1]{\mathbb{1}_{#1}}
\newcommand{\R}{\mathbb{R}}

%%% Solutions %%%

\newtheorem{solution}{Solution}
\newif\ifhidesolutions{}
% \hidesolutionstrue %decommenter pour cacher les SOLUTIONS

\ifhidesolutions{}
\usepackage{environ}
\NewEnviron{hide}{}
\let\solution\hide{}
\let\endsolution\endhide{}
\fi

%%% Title %%%

\title{PC 3 – Random vectors \& Convergence}
\author{}
\date{}

\begin{document}

\begin{flushleft}
  \textbf{Probability Refresher} \hfill \textbf{September 2023} \\
  \textbf{Master X-HEC} \hfill \\
  É\textbf{cole Polytechnique} \hfill
\end{flushleft}

{\let\newpage\relax\maketitle}
\vspace{-1.3cm}
\hrule

\vspace{0.5cm}
\begin{center}
  \section*{Set theory}
\end{center}

\begin{Exercise}
  For $n \geq 1$, let
  \[
    A_{n}=\left[-\frac{1}{n} ; 2+\frac{1}{n}\right], \quad B_{n}=\left[-\frac{5}{n} ; n^{2}\right] .
  \]
  \begin{enumerate}
    \item Compute $\bigcup_{n \geq 1} A_{n}, \bigcap_{n \geq 1} A_{n}$ and $\lim \sup _{n} A_{n}$, where $\lim \sup _{n} A_{n}$ is defined as
  \end{enumerate}
  \[
    \limsup A_{n}=\bigcap_{n \geq 1} \bigcup_{k \geq n} A_{k}=\left\{x \text { such that '' } x \in A_{n} \text { for infinitely many } n \text { '' }\right\} \text {. }
  \]
  \begin{enumerate}
    \setcounter{enumi}{1}
    \item Compute $\bigcup_{n \geq 1} B_{n}, \bigcap_{n \geq 1} B_{n}$ and $\lim \sup _{n} B_{n}$.

    \item Evaluate the following set

  \end{enumerate}
  \[
    \left\{x \text { such that } \sum_{n \geq 1} \mathbf{1}_{A_{n}}(x)=+\infty\right\} .
  \]
\end{Exercise}

\begin{solution}
  \begin{enumerate}
    \item
          Rappel de la d\'efinition: $\limsup_{n \rightarrow \infty} A_{n} =\cap_{k\geq 1} \cup_{n\ge k} A_n$.
          Il s'agit de l'\'ev\'enement o\`u:
          \[
            \omega \in \limsup_{n \rightarrow \infty} A_{n}  \ \Longleftrightarrow \  \textrm{il existe une infinit\'e de  $n$ tels que } \omega \in A_{n}
          \]

          La suite ${(A_n)}_{n\ge 1}$ \'etant monotone d\'ecroissante ($A_n \supset A_{n+1}$ pour tout $n\ge1$), on a pour tout $k\ge1$, $\cup_{n\ge k} A_n=[-1/k,3+1/k]$. D'une part, on voit que $[0,3]\subset A_k\subset \cup_{n\ge k} A_n$ pour tout $k$. D'autre part, pour tout $s<0$ et pour tout $t>3$ il existe $k$ tel que $s<-1/k$ et $t>3+1/k$. Donc, $\limsup_{n \rightarrow \infty} A_{n}=[0,3]$.
  \end{enumerate}
\end{solution}
\begin{center}
  \section*{Random vectors}
\end{center}

%%% Exercise 2 %%%

\begin{Exercise}
  Denote

  \[
    f(x, y)=c \mathrm{e}^{-x} \mathbb{1}_{|y| \leq x}
  \]

  \begin{enumerate}
    \item Find $c$ such that $f$ is a probability density function of a pair $(X, Y)$ of
          random variables.

    \item Compute the marginal distributions of $X$ and $Y$.

    \item Conclude on the independence of $X$ and $Y$.

  \end{enumerate}
\end{Exercise}

%%% Solution 2 %%%

\begin{solution}
  \begin{enumerate}
    \item We have:
          \begin{align*}
            f \text{ is a density } & \iff \int_{\R^2} c \mathrm{e}^{-x} \mathbb{1}_{|y| \leq x} d(x,y)= 1           \\
                                    & \iff \int_{\R} \int_{\R^+} c \mathrm{e}^{-x} \mathbb{1}_{|y| \leq x} dx dy = 1 \\
                                    & \iff \int_{\R^+} 2 x c \mathrm{e}^{-x} dx= 1                                   \\
                                    & \iff 2c \int_{\R^+} x \mathrm{e}^{-x} dx= 1                                    \\
                                    & \iff 2c = 1 \iff c = \frac{1}{2}
          \end{align*}
    \item Moreover,
          \begin{align*}
            f_X(x) & = x e^{-x}
          \end{align*}
          And
          \[
            f_Y(y) =  \frac{1}{2} e^{-y}
          \]
    \item We finally have:
          \[
            f(x, y) \neq f_X(x) f_Y(y)
          \]
          and the random variables therefore are not independents.
  \end{enumerate}
\end{solution}

%%% Exercise 3 %%%

\begin{Exercise}
  Let $X$ and $Y$ be two random variables taking their values in $\mathbb{N}$. Consider the joint probability mass function of $(X, Y)$ given by
  \[
    \proba{(X = i) \cap (Y = j)} = \frac{a}{2^{i+j}}, i, j \in \mathbb{N}, a \in \mathbb{R} .
  \]
  \begin{enumerate}
    \item Compute $a$.
    \item Give the marginal distributions of $X$ and $Y$.
    \item Are $X$ and $Y$ independent?
  \end{enumerate}
\end{Exercise}

%%% Solution 3 %%%

\begin{solution}
  \begin{enumerate}
    \item We have:
          \[
            \sum_{i, j = 0}^\infty \frac{a}{2^{i+j}} = a {\left( \sum_{i =
                  0}^\infty \frac{1}{2^i} \right)}^2 = a.2.2 = 4a
          \]
          Therefore, $4a = 1$ and finally $a = \frac{1}{4}$.
    \item We have:
          \begin{align*}
            \proba{X = i} & = \sum_{j = 0}^\infty \proba{(X = i) \cap (Y = j)} \\
                          & = \sum_{j = 0}^\infty \frac{1}{4.2^i.2^j}          \\
                          & = \frac{1}{2^{i+1}}
          \end{align*}
          In the same way:
          \[ \proba{Y = i} = \frac{1}{2^{i+1}} \]
    \item We have:
          \[
            \proba{(X = i) \cap (Y = j)} = \frac{1}{2^{i+j+2}} =
            \left(\frac{1}{2^{i+1}} \right) \left(\frac{1}{2^{j+1}} \right) = \proba{X = i}
            \proba{Y = j}
          \]
          And the random variables are therefore independents.
  \end{enumerate}
\end{solution}

%%% Exercise 4 %%%

\begin{Exercise}
  Denote

  \[
    f(x, y)=a\left(x^{2}+y^{2}\right) \mathbb{1}_{(x, y) \in {[-1,1]}^{2}} .
  \]

  \begin{enumerate}
    \item Find $a$ such that $f$ is a probability density. We denote $(X, Y)$ the pair of
          random variables with joint distribution $f$.

    \item Compute the marginal distributions of $X$ and $Y$.

    \item Compute the covariance of $X$ and $Y$.

    \item Are $X$ and $Y$ independent?

  \end{enumerate}
\end{Exercise}

%%% Solution 4 %%%

% To Do

\begin{solution}
  \begin{enumerate}
    \item We have
          \begin{align*}
            \int_{{[-1, 1]}^2} x^2 + y^2 dx dy & = 2 \int_{{[-1, 1]}^2} x^2 ~ dx ~ dy \\
                                               & = 4 \int_{{[-1, 1]}^2} x^2 ~ dx      \\
                                               & = 8 \int_0^1 x^2 ~ dx                \\
                                               & = \frac{8}{3}
          \end{align*}
          Therefore $a = \frac{3}{8}$.
    \item We have
          \begin{align*}
            f_X(x) & = a \int_{-1}^1 x^2 + y^2 dy                               \\
                   & = a \left( \int_{-1}^1 x^2 dy + \int_{-1}^1 y^2 dy \right) \\
                   & = 2 a (x^2 + \frac{1}{3})                                  \\
                   & = \frac{3}{4} (x^2 + \frac{1}{3})                          \\
                   & = \frac{3x^2 + 1}{4}
          \end{align*}
          And by symmetry $f_Y = f_X$.
    \item We have $$\esperance{X} = \int_{-1}^1 x \frac{3x^2 + 1}{4} dx = 0 $$ and
          \begin{align*}
            \cov{X} & = \esperance{X^2}                           \\
                    & = \int_{-1}^1 x^2  \frac{3x^2 + 1}{4} dx    \\
                    & = \frac{3}{4} \int_{-1}^1 x^4 + \frac{1}{2} \\
                    & = \frac{8}{10}
          \end{align*}
          and same for $Y$.
    \item As clearly $f \neq f_X \cdot f_Y$, $X$ and $Y$ are not independent.
  \end{enumerate}
\end{solution}

%%% Exercise 5 %%%

\begin{Exercise}
  Let $\mathbf{X}=\left(X_{1}, X_{2}, X_{3}\right)$ be a random vector with the following covariance matrix

  \[
    \operatorname{Cov}(\mathbf{X})=\left(\begin{array}{lll}
        2 & 1 & 3 \\
        1 & 5 & 6 \\
        3 & 6 & 9
      \end{array}\right)
  \]

  \begin{enumerate}
    \item Give the variance of $X_{2}$ and the covariance between $X_{1}$ and $X_{3}$.

    \item Compute the variance of $Z=X_{3}-\alpha_{1} X_{1}-\alpha_{2} X_{2}$ for
          $\alpha_{1}, \alpha_{2} \in \mathbb{R}$.

    \item Deduce that $X_{3}$ is almost surely a linear combination of $X_{1}$ and
          $X_{2}$.

    \item More generally, let $\mathbf{Y}$ be a random vector. Give a necessary and
          sufficient condition on the covariance matrix of $\mathbf{Y}$ ensuring that one
          of the components of $\mathbf{Y}$ is almost surely a linear combination of the
          components of $\mathbf{Y}$.

  \end{enumerate}

\end{Exercise}

%%% Solution 5 %%%

\begin{solution}
  \begin{enumerate}
    \item We have $\var{X_2} = 5$ and $\cov{X_1, X_3} = 3$.
    \item We want to compute $\var{Z}$. Let us note that $Z = X . y$ with $y = \begin{pmatrix} -1 \\ -1 \\ 1 \end{pmatrix}$. Therefore,
          \begin{align*}
            \var{Z} & = \var{X.y}     \\
                    & = y^T \var{X} y \\
                    & =
            \begin{pmatrix}
              -1 & -1 & 1
            \end{pmatrix}
            \begin{pmatrix}
              2 & 1 & 3 \\
              1 & 5 & 6 \\
              3 & 6 & 9
            \end{pmatrix}
            \begin{pmatrix}
              -1 \\ -1 \\ 1
            \end{pmatrix}            \\
                    & =
            \begin{pmatrix}
              -1 \\ -1 \\ 1
            \end{pmatrix}
            \begin{pmatrix}
              0 & 0 & 0
            \end{pmatrix}            \\
                    & = 0
          \end{align*}
          Thus, $\var{Z} = 0$.
    \item Thus, $Z = 0$ almost surely and finally $X_3$ is almost surely a linear combination of $X_1$ and $X_2$.
    \item The necessary and sufficient condition on $\cov{Y}$ is:
          \[
            \exists y\in \R^n, Y \cdot y = 0 \text{ almost surely} \iff \cov{Y} \text{ singular matrix}
          \]
          The proof is quite the same as in Question 2.:
          \begin{align*}
            \exists y\in \R^n, Y \cdot y = 0  \text{ almost surely} & \iff \var{Y \cdot y} = 0 \\
                                                                    & \iff y^T \var{Y} y = 0   \\
          \end{align*}
          As $\var{Y}$ is a symmetric matrix semi-definite positive, it can be written $\var{Y} = \Lambda^T \Lambda$ (Cholesky decomposition). Therefore :
          \begin{align*}
            \exists y\in \R^n, Y \cdot y = 0  \text{ almost surely} & \iff y^T \Lambda^T \Lambda y = 0     \\
                                                                    & \iff \|\Lambda y\|_2^2 = 0           \\
                                                                    & \iff \Lambda y = 0_n                 \\
                                                                    & \iff \Lambda^T \Lambda y = 0_n       \\
                                                                    & \iff \exists y, \var{Y} . y = 0_n    \\
                                                                    & \iff \cov{Y} \text{ singular matrix}
          \end{align*}
  \end{enumerate}
\end{solution}
\begin{center}
  \section*{Convergence}
\end{center}

\begin{Exercise}
  Let ${\left\{X_{i}\right\}}_{i \geq 0}$ be a sequence of i.i.d. Bernoulli variables with parameter $\theta$.
  \begin{enumerate}
    \item Show that $\sqrt{n}\left(\bar{X}_{n}-\theta\right)
            \stackrel{d}{\longrightarrow} \mathcal{N}(0, \theta(1-\theta))$, where
          $\bar{X}_{n}=n^{-1} \sum_{i=1}^{n} X_{i}$.

    \item Show that $\bar{X}_{n}\left(1-\bar{X}_{n}\right) \stackrel{P}{\longrightarrow}
            \theta(1-\theta)$.

    \item Show that $\sqrt{n}{\left(\bar{X}_{n}-\theta\right)}^{2}
            \stackrel{P}{\longrightarrow} 0$.

    \item Determine the limit distribution of
          $\sqrt{n}\left(\bar{X}_{n}\left(1-\bar{X}_{n}\right)-\theta(1-\theta)\right)$.

  \end{enumerate}
\end{Exercise}

\begin{solution}
  \begin{enumerate}
    \item Par le TCL, on a $\sqrt{n} ( \bar{X}_n - \theta)=\sqrt{n} ( \bar{X}_n -
            \esperance{}[X_1]) \overset{d}{\rightarrow} \mathcal N(0,Var(X_1))=\mathcal N
            (0,\theta(1-\theta))$.
    \item Par la LGN, on a $\bar{X}_n \overset{P}{\rightarrow}\esperance{}[X_1]=\theta$.
          La fonction $h(x)=x(1-x)$ \'etant continue, on obtient par le th\'eor\`eme de
          continuit\'e, $\bar{X}_n(1-\bar{X}_n)=h(\bar{X}_n)\overset{P}{\rightarrow}
            h(\theta)=\theta(1-\theta)$.
    \item On a
          \[
            \sqrt{n} ( \bar{X}_n - \theta)^2= \underbrace{\sqrt{n} ( \bar{X}_n -
              \theta)}_{\overset{d}{\rightarrow} \mathcal N
              (0,\theta(1-\theta))}\underbrace{( \bar{X}_n -
              \theta)}_{\overset{P}{\rightarrow} 0} \overset{d}{\rightarrow} 0\times \mathcal
            N (0,\theta(1-\theta))=0
          \]
          La convergence en loi vers une constante est \'equivalente \`a la convergence
          en probabilit\'e, d'o\`u le r\'esultat.
    \item On \'ecrit
          \begin{align*}
            \sqrt{n} \left( \bar{X}_n (1 - \bar{X}_n) - \theta(1-\theta) \right)
             & = \sqrt{n} \left( (\bar{X}_n-\theta) (1 - \bar{X}_n)+\theta(1 - \bar{X}_n) - \theta(1-\theta) \right)                                                                            \\
             & = \sqrt{n} \left( (\bar{X}_n-\theta) (1 - \bar{X}_n)-\theta( \bar{X}_n -\theta) \right)                                                                                          \\
             & = \underbrace{\sqrt{n}   (\bar{X}_n-\theta)}_{\overset{d}{\rightarrow} \mathcal N (0,\theta(1-\theta))} \underbrace{(1 - \bar{X}_n-\theta)}_{\overset{P}{\rightarrow} 1-2\theta} \\
             & \overset{d}{\rightarrow} (1-2\theta) \mathcal N (0,\theta(1-\theta))= \mathcal N (0,{(1-2\theta)}^2\theta(1-\theta)),
          \end{align*}
          par le lemme de Slutsky.
  \end{enumerate}
\end{solution}

\begin{Exercise}
  Let ${\left(X_{n}\right)}_{n \geq 1}$ be a sequence of i.i.d.\,square-integrable random variables with mean $m$ and variance $\sigma^{2}>0$. Denote $\bar{X}_{n}=\frac{1}{n} \sum_{i=1}^{n} X_{i}$ and $\hat{\sigma}_{n}^{2}=\frac{1}{n} \sum_{i=1}^{n}{\left(X_{i}-\bar{X}_{n}\right)}^{2}$.

  \begin{enumerate}
    \item Show that $\hat{\sigma}_{n}^{2}$ converges in probability to $\sigma^{2}$ as $n
            \rightarrow \infty$.

    \item Determine the limit distribution of $\sqrt{n}\left(\bar{X}_{n}-m\right) /
            \hat{\sigma}_{n}$.

  \end{enumerate}
\end{Exercise}

\begin{solution}
  Commen\c cons par \'etudier le comportement limite de $\hat{\sigma}_n^2$ quand $n\to +\infty$.
  \begin{align*}
    (n-1)\hat{\sigma}_n^2
     & =\sum_{k=1}^n{(X_k-\bar X_n)}^2                                                  \\
     & = \sum_{k=1}^n{(X_k-m)}^2 + 2\sum_{k=1}^n(X_k-m)(m-\bar X_n) + n{(m-\bar X_n)}^2 \\
     & = \sum_{k=1}^n{(X_k-m)}^2 - n{(m-\bar X_n)}^2.
  \end{align*}
  Donc
  \begin{align*}
    \frac{n-1}{n}\hat{\sigma}_n^2 & = \frac{1}{n}\sum_{k=1}^n(X_k-m)^2 - {(m-\bar X_n)}^2                           \\
                                  & \overset{p.s.}{\longrightarrow}\esperance{}[{(X_1-m)}^2]-0= \mathrm{Var}(X_1)=:
    \sigma^2,
  \end{align*}
  o\`u la limite est donn\'ee par la loi des grands nombres. Par suite, $\hat{\sigma}_n \to \sigma$ presque s\^urement. Notons $Z_n = \sqrt{n}(\bar X_n-m)$ qui converge en loi, d'apr\`es le th\'eor\`eme limite central vers une variable al\'eatoire gaussienne $Z\sim \mathcal{N}(0,\sigma^2)$. D'apr\`es le lemme de Slutsky, le couple $(Z_n,\hat{\sigma}_n^{-1})$ converge en loi vers $(Z,\sigma^{-1})$. En particulier, la fonction produit \'etant continue, $\frac{Z_n}{\hat{\sigma}_n} \overset{d}{\rightarrow} Z/\sigma \sim \mathcal{N}(0,1)$
\end{solution}

\begin{Exercise}
  (Poisson model). Let $\left(X_{1}, \ldots, X_{n}\right)$ be an i.i.d. sample from the Poisson distribution with unknown parameter $\lambda>0$. Denote $\bar{X}_{n}=\frac{1}{n} \sum_{i=1}^{n} X_{i}$.

  \begin{enumerate}
    \item Show that $\bar{X}_{n}$ is an unbiased estimator of $\lambda$, that is
          $\mathbb{E}\left[\bar{X}_{n}\right]=\lambda$.

    \item Show that $\bar{X}_{n}$ converges in probability to $\lambda$ when $n$ tends to
          infinity.

    \item Determine the limit distribution of $\sqrt{n}\left(\bar{X}_{n}-\lambda\right) /
            \sqrt{\bar{X}_{n}}$.

    \item Find an appropriate function $g$ such that
          $\sqrt{n}\left(g\left(\bar{X}_{n}\right)-g(\lambda)\right)
            \stackrel{d}{\longrightarrow} \mathcal{N}(0,1)$.

  \end{enumerate}
\end{Exercise}

\begin{solution}
  \begin{enumerate}
    \item On rappelle, pour $X\sim\mathcal P(\lambda)$, $\lambda>0$, on a
          $\esperance{}[X]=\var(X)=\lambda$. Alors l'estimateur $\bar X_n$ est alors sans
          biais ($\esperance{}[\bar X_n] =\lambda$), consistant en vertu de la LFGN
          ($\bar X_n\longrightarrow \esperance{}[X_1]=\lambda~p.s$), et enfin $\bar{X}_n$
          est asymptotiquement normal par le TCL:
          \[
            \sqrt n( \bar{X}
            _n-\lambda)=\sqrt n( \bar X_n-\esperance{}[X_1])\stackrel{\mathcal
              L}\longrightarrow\mathcal N(0,\var(X_1))=\mathcal
            N(0,\lambda),\quad\text{lorsque }n\to\infty
          \]

    \item En utilisant la question a et le lemme de Slutsky, on obtient
          \begin{align*}
            \sqrt n \left(\frac{\bar X_n-\lambda}{\sqrt{\bar X_n}} \right)
             & = \underbrace{ \sqrt n \left(\frac{\bar X_n-\lambda}{\sqrt{\lambda}} \right) }_{\stackrel{\mathcal L}\longrightarrow\mathcal N(0,1)} \underbrace{\frac{\sqrt{\lambda}}{\sqrt{\bar X_n}}}_{\stackrel{P}\longrightarrow \frac{\sqrt{\lambda}}{\sqrt{\esperance{}[X_1]}}=1}
            \stackrel{\mathcal L}\longrightarrow\mathcal N(0,1).
          \end{align*}
    \item D'apr\`es la delta m\'ethode, pour toute fonction $g$ continument d\'erivable
          sur $\R_+$, on a
          \[
            \sqrt n \left(g(\bar X_n)-g(\lambda)
            \right)\stackrel{\mathcal L}\longrightarrow\mathcal
            N(0,{(g'(\lambda))}^2\var(X))
          \]
          Nous cherchons donc une fonction $g$ telle que la variance limite vaut 1. Ce
          qui veut dire
          \[
            {(g'(\lambda))}^2\var(X)=1
            \Leftrightarrow {(g'(\lambda))}^2 = \frac1\lambda
          \]
          On peut alors choisir $g(u) = 2\sqrt u$ avec d\'eriv\'ee $g'(u)= 1/\sqrt u$ et
          on obtient
          \[
            \sqrt n
            \left(2\sqrt{\bar X_n}-2\sqrt \lambda \right)\stackrel{\mathcal
              L}\longrightarrow\mathcal
            N\left(0,{\left(\frac1{\sqrt\lambda}\right)}^2\lambda\right)=\mathcal N(0,1).
          \]

  \end{enumerate}

\end{solution}

\begin{Exercise}
  Define the random variable

  \[
    Y=\mathbb{1}\{\theta>X\}
  \]

  where $\theta \in \mathbb{R}$ and $X$ is a random variable with standard normal
  distribution $\mathcal{N}(0,1)$. We observe a sample $Y_{1}, \ldots, Y_{n}$ of
  i.i.d.\,realizations of $Y$ and suppose that parameter $\theta$ is unknown.
  Denote by $\Phi$ the cumulative distribution function of the standard normal
  distribution $\mathcal{N}(0,1)$. An estimator $\hat{\theta}_{n}$ of $\theta$ is
  given by

  \[
    \hat{\theta}_{n}=\Phi^{-1}\left(\bar{Y}_{n}\right)
  \]

  where $\bar{Y}_{n}=\frac{1}{n} \sum_{i=1}^{n} Y_{i}$

  \begin{enumerate}
    \item Determine the distribution of $Y$.

    \item Study the convergence in probability of $\hat{\theta}_{n}$ towards $\theta$
          when $n$ tends to infinity.

    \item Study the limit distribution of $\sqrt{n}\left(\hat{\theta}_{n}-\theta\right)$.

  \end{enumerate}
\end{Exercise}

\begin{solution}
  \begin{enumerate}
    \item
          Comme $Y$ prend ses valeurs dans $\{0,1\}$, $Y$ suit une loi de Bernoulli avec
          param\`etre $\proba{Y=1} = \proba{\theta>\xi} = \Phi(\theta)$

    \item
          Puisque $\frac1n\sum_{i=1}^n Y_i\to\esperance{}[Y_1]=\Phi(\theta)$ p.s et $\Phi^{-1}$ est
          une fonction continue, on a $\hat \theta_n=\Phi^{-1}(\frac1n\sum_{i=1}^n Y_i)\to
            \Phi^{-1}(\Phi(\theta))=\theta~p.s$. Donc $\hat\theta_n$ est consistant pour
          $\theta$.

    \item
          En vertu du TCL (car $\esperance{}[Y^2_1]<\infty$), on a $\sqrt
            n(\frac1n\sum_{i=1}^n Y_i-\Phi(\theta))\stackrel{\mathcal
              L}\longrightarrow\mathcal N(0,\var(Y_1))=\mathcal
            N(0,\Phi(\theta)(1-\Phi(\theta)))$. La fonction $\Phi^{-1}(\theta)$ est
          continument d\'erivable avec d\'eriv\'ee $(\Phi^{-1})'(\theta) =
            1/\varphi(\Phi^{-1}(\theta))$. On obtient par la delta-m\'ethode
          \begin{align*}
            \sqrt n(\hat\theta_n-\theta) =\sqrt n\left(\Phi^{-1}\left(\frac1n\sum_{i=1}^n Y_i\right)-\Phi^{-1}(\Phi(\theta))\right)
             & \stackrel{\mathcal L}\longrightarrow \mathcal N(0, {((\Phi^{-1})'(\Phi(\theta)))}^2   \Phi(\theta)(1-\Phi(\theta))) \\
             & =\mathcal N\left(0, \frac{  \Phi(\theta)(1-\Phi(\theta))}{ \varphi^2(\theta)}\right)
          \end{align*}

  \end{enumerate}

\end{solution}
% \begin{center}
  \section*{Convergence}
\end{center}

\begin{Exercise}
  Let ${\left\{X_{i}\right\}}_{i \geq 0}$ be a sequence of i.i.d. Bernoulli variables with parameter $\theta$.
  \begin{enumerate}
    \item Show that $\sqrt{n}\left(\bar{X}_{n}-\theta\right)
            \stackrel{d}{\longrightarrow} \mathcal{N}(0, \theta(1-\theta))$, where
          $\bar{X}_{n}=n^{-1} \sum_{i=1}^{n} X_{i}$.

    \item Show that $\bar{X}_{n}\left(1-\bar{X}_{n}\right) \stackrel{P}{\longrightarrow}
            \theta(1-\theta)$.

    \item Show that $\sqrt{n}{\left(\bar{X}_{n}-\theta\right)}^{2}
            \stackrel{P}{\longrightarrow} 0$.

    \item Determine the limit distribution of
          $\sqrt{n}\left(\bar{X}_{n}\left(1-\bar{X}_{n}\right)-\theta(1-\theta)\right)$.

  \end{enumerate}
\end{Exercise}

\begin{solution}
  \begin{enumerate}
    \item Par le TCL, on a $\sqrt{n} ( \bar{X}_n - \theta)=\sqrt{n} ( \bar{X}_n -
            \esperance{}[X_1]) \overset{d}{\rightarrow} \mathcal N(0,Var(X_1))=\mathcal N
            (0,\theta(1-\theta))$.
    \item Par la LGN, on a $\bar{X}_n \overset{P}{\rightarrow}\esperance{}[X_1]=\theta$.
          La fonction $h(x)=x(1-x)$ \'etant continue, on obtient par le th\'eor\`eme de
          continuit\'e, $\bar{X}_n(1-\bar{X}_n)=h(\bar{X}_n)\overset{P}{\rightarrow}
            h(\theta)=\theta(1-\theta)$.
    \item On a
          \[
            \sqrt{n} ( \bar{X}_n - \theta)^2= \underbrace{\sqrt{n} ( \bar{X}_n -
              \theta)}_{\overset{d}{\rightarrow} \mathcal N
              (0,\theta(1-\theta))}\underbrace{( \bar{X}_n -
              \theta)}_{\overset{P}{\rightarrow} 0} \overset{d}{\rightarrow} 0\times \mathcal
            N (0,\theta(1-\theta))=0
          \]
          La convergence en loi vers une constante est \'equivalente \`a la convergence
          en probabilit\'e, d'o\`u le r\'esultat.
    \item On \'ecrit
          \begin{align*}
            \sqrt{n} \left( \bar{X}_n (1 - \bar{X}_n) - \theta(1-\theta) \right)
             & = \sqrt{n} \left( (\bar{X}_n-\theta) (1 - \bar{X}_n)+\theta(1 - \bar{X}_n) - \theta(1-\theta) \right)                                                                            \\
             & = \sqrt{n} \left( (\bar{X}_n-\theta) (1 - \bar{X}_n)-\theta( \bar{X}_n -\theta) \right)                                                                                          \\
             & = \underbrace{\sqrt{n}   (\bar{X}_n-\theta)}_{\overset{d}{\rightarrow} \mathcal N (0,\theta(1-\theta))} \underbrace{(1 - \bar{X}_n-\theta)}_{\overset{P}{\rightarrow} 1-2\theta} \\
             & \overset{d}{\rightarrow} (1-2\theta) \mathcal N (0,\theta(1-\theta))= \mathcal N (0,{(1-2\theta)}^2\theta(1-\theta)),
          \end{align*}
          par le lemme de Slutsky.
  \end{enumerate}
\end{solution}

\begin{Exercise}
  Let ${\left(X_{n}\right)}_{n \geq 1}$ be a sequence of i.i.d.\,square-integrable random variables with mean $m$ and variance $\sigma^{2}>0$. Denote $\bar{X}_{n}=\frac{1}{n} \sum_{i=1}^{n} X_{i}$ and $\hat{\sigma}_{n}^{2}=\frac{1}{n} \sum_{i=1}^{n}{\left(X_{i}-\bar{X}_{n}\right)}^{2}$.

  \begin{enumerate}
    \item Show that $\hat{\sigma}_{n}^{2}$ converges in probability to $\sigma^{2}$ as $n
            \rightarrow \infty$.

    \item Determine the limit distribution of $\sqrt{n}\left(\bar{X}_{n}-m\right) /
            \hat{\sigma}_{n}$.

  \end{enumerate}
\end{Exercise}

\begin{solution}
  Commen\c cons par \'etudier le comportement limite de $\hat{\sigma}_n^2$ quand $n\to +\infty$.
  \begin{align*}
    (n-1)\hat{\sigma}_n^2
     & =\sum_{k=1}^n{(X_k-\bar X_n)}^2                                                  \\
     & = \sum_{k=1}^n{(X_k-m)}^2 + 2\sum_{k=1}^n(X_k-m)(m-\bar X_n) + n{(m-\bar X_n)}^2 \\
     & = \sum_{k=1}^n{(X_k-m)}^2 - n{(m-\bar X_n)}^2.
  \end{align*}
  Donc
  \begin{align*}
    \frac{n-1}{n}\hat{\sigma}_n^2 & = \frac{1}{n}\sum_{k=1}^n(X_k-m)^2 - {(m-\bar X_n)}^2                           \\
                                  & \overset{p.s.}{\longrightarrow}\esperance{}[{(X_1-m)}^2]-0= \mathrm{Var}(X_1)=:
    \sigma^2,
  \end{align*}
  o\`u la limite est donn\'ee par la loi des grands nombres. Par suite, $\hat{\sigma}_n \to \sigma$ presque s\^urement. Notons $Z_n = \sqrt{n}(\bar X_n-m)$ qui converge en loi, d'apr\`es le th\'eor\`eme limite central vers une variable al\'eatoire gaussienne $Z\sim \mathcal{N}(0,\sigma^2)$. D'apr\`es le lemme de Slutsky, le couple $(Z_n,\hat{\sigma}_n^{-1})$ converge en loi vers $(Z,\sigma^{-1})$. En particulier, la fonction produit \'etant continue, $\frac{Z_n}{\hat{\sigma}_n} \overset{d}{\rightarrow} Z/\sigma \sim \mathcal{N}(0,1)$
\end{solution}

\begin{Exercise}
  (Poisson model). Let $\left(X_{1}, \ldots, X_{n}\right)$ be an i.i.d. sample from the Poisson distribution with unknown parameter $\lambda>0$. Denote $\bar{X}_{n}=\frac{1}{n} \sum_{i=1}^{n} X_{i}$.

  \begin{enumerate}
    \item Show that $\bar{X}_{n}$ is an unbiased estimator of $\lambda$, that is
          $\mathbb{E}\left[\bar{X}_{n}\right]=\lambda$.

    \item Show that $\bar{X}_{n}$ converges in probability to $\lambda$ when $n$ tends to
          infinity.

    \item Determine the limit distribution of $\sqrt{n}\left(\bar{X}_{n}-\lambda\right) /
            \sqrt{\bar{X}_{n}}$.

    \item Find an appropriate function $g$ such that
          $\sqrt{n}\left(g\left(\bar{X}_{n}\right)-g(\lambda)\right)
            \stackrel{d}{\longrightarrow} \mathcal{N}(0,1)$.

  \end{enumerate}
\end{Exercise}

\begin{solution}
  \begin{enumerate}
    \item On rappelle, pour $X\sim\mathcal P(\lambda)$, $\lambda>0$, on a
          $\esperance{}[X]=\var(X)=\lambda$. Alors l'estimateur $\bar X_n$ est alors sans
          biais ($\esperance{}[\bar X_n] =\lambda$), consistant en vertu de la LFGN
          ($\bar X_n\longrightarrow \esperance{}[X_1]=\lambda~p.s$), et enfin $\bar{X}_n$
          est asymptotiquement normal par le TCL:
          \[
            \sqrt n( \bar{X}
            _n-\lambda)=\sqrt n( \bar X_n-\esperance{}[X_1])\stackrel{\mathcal
              L}\longrightarrow\mathcal N(0,\var(X_1))=\mathcal
            N(0,\lambda),\quad\text{lorsque }n\to\infty
          \]

    \item En utilisant la question a et le lemme de Slutsky, on obtient
          \begin{align*}
            \sqrt n \left(\frac{\bar X_n-\lambda}{\sqrt{\bar X_n}} \right)
             & = \underbrace{ \sqrt n \left(\frac{\bar X_n-\lambda}{\sqrt{\lambda}} \right) }_{\stackrel{\mathcal L}\longrightarrow\mathcal N(0,1)} \underbrace{\frac{\sqrt{\lambda}}{\sqrt{\bar X_n}}}_{\stackrel{P}\longrightarrow \frac{\sqrt{\lambda}}{\sqrt{\esperance{}[X_1]}}=1}
            \stackrel{\mathcal L}\longrightarrow\mathcal N(0,1).
          \end{align*}
    \item D'apr\`es la delta m\'ethode, pour toute fonction $g$ continument d\'erivable
          sur $\R_+$, on a
          \[
            \sqrt n \left(g(\bar X_n)-g(\lambda)
            \right)\stackrel{\mathcal L}\longrightarrow\mathcal
            N(0,{(g'(\lambda))}^2\var(X))
          \]
          Nous cherchons donc une fonction $g$ telle que la variance limite vaut 1. Ce
          qui veut dire
          \[
            {(g'(\lambda))}^2\var(X)=1
            \Leftrightarrow {(g'(\lambda))}^2 = \frac1\lambda
          \]
          On peut alors choisir $g(u) = 2\sqrt u$ avec d\'eriv\'ee $g'(u)= 1/\sqrt u$ et
          on obtient
          \[
            \sqrt n
            \left(2\sqrt{\bar X_n}-2\sqrt \lambda \right)\stackrel{\mathcal
              L}\longrightarrow\mathcal
            N\left(0,{\left(\frac1{\sqrt\lambda}\right)}^2\lambda\right)=\mathcal N(0,1).
          \]

  \end{enumerate}

\end{solution}

\begin{Exercise}
  Define the random variable

  \[
    Y=\mathbb{1}\{\theta>X\}
  \]

  where $\theta \in \mathbb{R}$ and $X$ is a random variable with standard normal
  distribution $\mathcal{N}(0,1)$. We observe a sample $Y_{1}, \ldots, Y_{n}$ of
  i.i.d.\,realizations of $Y$ and suppose that parameter $\theta$ is unknown.
  Denote by $\Phi$ the cumulative distribution function of the standard normal
  distribution $\mathcal{N}(0,1)$. An estimator $\hat{\theta}_{n}$ of $\theta$ is
  given by

  \[
    \hat{\theta}_{n}=\Phi^{-1}\left(\bar{Y}_{n}\right)
  \]

  where $\bar{Y}_{n}=\frac{1}{n} \sum_{i=1}^{n} Y_{i}$

  \begin{enumerate}
    \item Determine the distribution of $Y$.

    \item Study the convergence in probability of $\hat{\theta}_{n}$ towards $\theta$
          when $n$ tends to infinity.

    \item Study the limit distribution of $\sqrt{n}\left(\hat{\theta}_{n}-\theta\right)$.

  \end{enumerate}
\end{Exercise}

\begin{solution}
  \begin{enumerate}
    \item
          Comme $Y$ prend ses valeurs dans $\{0,1\}$, $Y$ suit une loi de Bernoulli avec
          param\`etre $\proba{Y=1} = \proba{\theta>\xi} = \Phi(\theta)$

    \item
          Puisque $\frac1n\sum_{i=1}^n Y_i\to\esperance{}[Y_1]=\Phi(\theta)$ p.s et $\Phi^{-1}$ est
          une fonction continue, on a $\hat \theta_n=\Phi^{-1}(\frac1n\sum_{i=1}^n Y_i)\to
            \Phi^{-1}(\Phi(\theta))=\theta~p.s$. Donc $\hat\theta_n$ est consistant pour
          $\theta$.

    \item
          En vertu du TCL (car $\esperance{}[Y^2_1]<\infty$), on a $\sqrt
            n(\frac1n\sum_{i=1}^n Y_i-\Phi(\theta))\stackrel{\mathcal
              L}\longrightarrow\mathcal N(0,\var(Y_1))=\mathcal
            N(0,\Phi(\theta)(1-\Phi(\theta)))$. La fonction $\Phi^{-1}(\theta)$ est
          continument d\'erivable avec d\'eriv\'ee $(\Phi^{-1})'(\theta) =
            1/\varphi(\Phi^{-1}(\theta))$. On obtient par la delta-m\'ethode
          \begin{align*}
            \sqrt n(\hat\theta_n-\theta) =\sqrt n\left(\Phi^{-1}\left(\frac1n\sum_{i=1}^n Y_i\right)-\Phi^{-1}(\Phi(\theta))\right)
             & \stackrel{\mathcal L}\longrightarrow \mathcal N(0, {((\Phi^{-1})'(\Phi(\theta)))}^2   \Phi(\theta)(1-\Phi(\theta))) \\
             & =\mathcal N\left(0, \frac{  \Phi(\theta)(1-\Phi(\theta))}{ \varphi^2(\theta)}\right)
          \end{align*}

  \end{enumerate}

\end{solution}

\end{document}