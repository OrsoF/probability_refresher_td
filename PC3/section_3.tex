\begin{center}
  \section*{Convergence}
\end{center}

\begin{Exercise}
  Let ${\left\{X_{i}\right\}}_{i \geq 0}$ be a sequence of i.i.d. Bernoulli variables with parameter $\theta$.
  \begin{enumerate}
    \item Show that $\sqrt{n}\left(\bar{X}_{n}-\theta\right)
            \stackrel{d}{\longrightarrow} \mathcal{N}(0, \theta(1-\theta))$, where
          $\bar{X}_{n}=n^{-1} \sum_{i=1}^{n} X_{i}$.

    \item Show that $\bar{X}_{n}\left(1-\bar{X}_{n}\right) \stackrel{P}{\longrightarrow}
            \theta(1-\theta)$.

    \item Show that $\sqrt{n}{\left(\bar{X}_{n}-\theta\right)}^{2}
            \stackrel{P}{\longrightarrow} 0$.

    \item Determine the limit distribution of
          $\sqrt{n}\left(\bar{X}_{n}\left(1-\bar{X}_{n}\right)-\theta(1-\theta)\right)$.

  \end{enumerate}
\end{Exercise}

\begin{solution}
  \begin{enumerate}
    \item Par le TCL, on a $\sqrt{n} ( \bar{X}_n - \theta)=\sqrt{n} ( \bar{X}_n -
            \esperance{}[X_1]) \overset{d}{\rightarrow} \mathcal N(0,Var(X_1))=\mathcal N
            (0,\theta(1-\theta))$.
    \item Par la LGN, on a $\bar{X}_n \overset{P}{\rightarrow}\esperance{}[X_1]=\theta$.
          La fonction $h(x)=x(1-x)$ \'etant continue, on obtient par le th\'eor\`eme de
          continuit\'e, $\bar{X}_n(1-\bar{X}_n)=h(\bar{X}_n)\overset{P}{\rightarrow}
            h(\theta)=\theta(1-\theta)$.
    \item On a
          \[
            \sqrt{n} ( \bar{X}_n - \theta)^2= \underbrace{\sqrt{n} ( \bar{X}_n -
              \theta)}_{\overset{d}{\rightarrow} \mathcal N
              (0,\theta(1-\theta))}\underbrace{( \bar{X}_n -
              \theta)}_{\overset{P}{\rightarrow} 0} \overset{d}{\rightarrow} 0\times \mathcal
            N (0,\theta(1-\theta))=0
          \]
          La convergence en loi vers une constante est \'equivalente \`a la convergence
          en probabilit\'e, d'o\`u le r\'esultat.
    \item On \'ecrit
          \begin{align*}
            \sqrt{n} \left( \bar{X}_n (1 - \bar{X}_n) - \theta(1-\theta) \right)
             & = \sqrt{n} \left( (\bar{X}_n-\theta) (1 - \bar{X}_n)+\theta(1 - \bar{X}_n) - \theta(1-\theta) \right)                                                                            \\
             & = \sqrt{n} \left( (\bar{X}_n-\theta) (1 - \bar{X}_n)-\theta( \bar{X}_n -\theta) \right)                                                                                          \\
             & = \underbrace{\sqrt{n}   (\bar{X}_n-\theta)}_{\overset{d}{\rightarrow} \mathcal N (0,\theta(1-\theta))} \underbrace{(1 - \bar{X}_n-\theta)}_{\overset{P}{\rightarrow} 1-2\theta} \\
             & \overset{d}{\rightarrow} (1-2\theta) \mathcal N (0,\theta(1-\theta))= \mathcal N (0,{(1-2\theta)}^2\theta(1-\theta)),
          \end{align*}
          par le lemme de Slutsky.
  \end{enumerate}
\end{solution}

\begin{Exercise}
  Let ${\left(X_{n}\right)}_{n \geq 1}$ be a sequence of i.i.d.\,square-integrable random variables with mean $m$ and variance $\sigma^{2}>0$. Denote $\bar{X}_{n}=\frac{1}{n} \sum_{i=1}^{n} X_{i}$ and $\hat{\sigma}_{n}^{2}=\frac{1}{n} \sum_{i=1}^{n}{\left(X_{i}-\bar{X}_{n}\right)}^{2}$.

  \begin{enumerate}
    \item Show that $\hat{\sigma}_{n}^{2}$ converges in probability to $\sigma^{2}$ as $n
            \rightarrow \infty$.

    \item Determine the limit distribution of $\sqrt{n}\left(\bar{X}_{n}-m\right) /
            \hat{\sigma}_{n}$.

  \end{enumerate}
\end{Exercise}

\begin{solution}
  Commen\c cons par \'etudier le comportement limite de $\hat{\sigma}_n^2$ quand $n\to +\infty$.
  \begin{align*}
    (n-1)\hat{\sigma}_n^2
     & =\sum_{k=1}^n{(X_k-\bar X_n)}^2                                                  \\
     & = \sum_{k=1}^n{(X_k-m)}^2 + 2\sum_{k=1}^n(X_k-m)(m-\bar X_n) + n{(m-\bar X_n)}^2 \\
     & = \sum_{k=1}^n{(X_k-m)}^2 - n{(m-\bar X_n)}^2.
  \end{align*}
  Donc
  \begin{align*}
    \frac{n-1}{n}\hat{\sigma}_n^2 & = \frac{1}{n}\sum_{k=1}^n(X_k-m)^2 - {(m-\bar X_n)}^2                           \\
                                  & \overset{p.s.}{\longrightarrow}\esperance{}[{(X_1-m)}^2]-0= \mathrm{Var}(X_1)=:
    \sigma^2,
  \end{align*}
  o\`u la limite est donn\'ee par la loi des grands nombres. Par suite, $\hat{\sigma}_n \to \sigma$ presque s\^urement. Notons $Z_n = \sqrt{n}(\bar X_n-m)$ qui converge en loi, d'apr\`es le th\'eor\`eme limite central vers une variable al\'eatoire gaussienne $Z\sim \mathcal{N}(0,\sigma^2)$. D'apr\`es le lemme de Slutsky, le couple $(Z_n,\hat{\sigma}_n^{-1})$ converge en loi vers $(Z,\sigma^{-1})$. En particulier, la fonction produit \'etant continue, $\frac{Z_n}{\hat{\sigma}_n} \overset{d}{\rightarrow} Z/\sigma \sim \mathcal{N}(0,1)$
\end{solution}

\begin{Exercise}
  (Poisson model). Let $\left(X_{1}, \ldots, X_{n}\right)$ be an i.i.d. sample from the Poisson distribution with unknown parameter $\lambda>0$. Denote $\bar{X}_{n}=\frac{1}{n} \sum_{i=1}^{n} X_{i}$.

  \begin{enumerate}
    \item Show that $\bar{X}_{n}$ is an unbiased estimator of $\lambda$, that is
          $\mathbb{E}\left[\bar{X}_{n}\right]=\lambda$.

    \item Show that $\bar{X}_{n}$ converges in probability to $\lambda$ when $n$ tends to
          infinity.

    \item Determine the limit distribution of $\sqrt{n}\left(\bar{X}_{n}-\lambda\right) /
            \sqrt{\bar{X}_{n}}$.

    \item Find an appropriate function $g$ such that
          $\sqrt{n}\left(g\left(\bar{X}_{n}\right)-g(\lambda)\right)
            \stackrel{d}{\longrightarrow} \mathcal{N}(0,1)$.

  \end{enumerate}
\end{Exercise}

\begin{solution}
  \begin{enumerate}
    \item On rappelle, pour $X\sim\mathcal P(\lambda)$, $\lambda>0$, on a
          $\esperance{}[X]=\var(X)=\lambda$. Alors l'estimateur $\bar X_n$ est alors sans
          biais ($\esperance{}[\bar X_n] =\lambda$), consistant en vertu de la LFGN
          ($\bar X_n\longrightarrow \esperance{}[X_1]=\lambda~p.s$), et enfin $\bar{X}_n$
          est asymptotiquement normal par le TCL:
          \[
            \sqrt n( \bar{X}
            _n-\lambda)=\sqrt n( \bar X_n-\esperance{}[X_1])\stackrel{\mathcal
              L}\longrightarrow\mathcal N(0,\var(X_1))=\mathcal
            N(0,\lambda),\quad\text{lorsque }n\to\infty
          \]

    \item En utilisant la question a et le lemme de Slutsky, on obtient
          \begin{align*}
            \sqrt n \left(\frac{\bar X_n-\lambda}{\sqrt{\bar X_n}} \right)
             & = \underbrace{ \sqrt n \left(\frac{\bar X_n-\lambda}{\sqrt{\lambda}} \right) }_{\stackrel{\mathcal L}\longrightarrow\mathcal N(0,1)} \underbrace{\frac{\sqrt{\lambda}}{\sqrt{\bar X_n}}}_{\stackrel{P}\longrightarrow \frac{\sqrt{\lambda}}{\sqrt{\esperance{}[X_1]}}=1}
            \stackrel{\mathcal L}\longrightarrow\mathcal N(0,1).
          \end{align*}
    \item D'apr\`es la delta m\'ethode, pour toute fonction $g$ continument d\'erivable
          sur $\R_+$, on a
          \[
            \sqrt n \left(g(\bar X_n)-g(\lambda)
            \right)\stackrel{\mathcal L}\longrightarrow\mathcal
            N(0,{(g'(\lambda))}^2\var(X))
          \]
          Nous cherchons donc une fonction $g$ telle que la variance limite vaut 1. Ce
          qui veut dire
          \[
            {(g'(\lambda))}^2\var(X)=1
            \Leftrightarrow {(g'(\lambda))}^2 = \frac1\lambda
          \]
          On peut alors choisir $g(u) = 2\sqrt u$ avec d\'eriv\'ee $g'(u)= 1/\sqrt u$ et
          on obtient
          \[
            \sqrt n
            \left(2\sqrt{\bar X_n}-2\sqrt \lambda \right)\stackrel{\mathcal
              L}\longrightarrow\mathcal
            N\left(0,{\left(\frac1{\sqrt\lambda}\right)}^2\lambda\right)=\mathcal N(0,1).
          \]

  \end{enumerate}

\end{solution}

\begin{Exercise}
  Define the random variable

  \[
    Y=\mathbb{1}\{\theta>X\}
  \]

  where $\theta \in \mathbb{R}$ and $X$ is a random variable with standard normal
  distribution $\mathcal{N}(0,1)$. We observe a sample $Y_{1}, \ldots, Y_{n}$ of
  i.i.d.\,realizations of $Y$ and suppose that parameter $\theta$ is unknown.
  Denote by $\Phi$ the cumulative distribution function of the standard normal
  distribution $\mathcal{N}(0,1)$. An estimator $\hat{\theta}_{n}$ of $\theta$ is
  given by

  \[
    \hat{\theta}_{n}=\Phi^{-1}\left(\bar{Y}_{n}\right)
  \]

  where $\bar{Y}_{n}=\frac{1}{n} \sum_{i=1}^{n} Y_{i}$

  \begin{enumerate}
    \item Determine the distribution of $Y$.

    \item Study the convergence in probability of $\hat{\theta}_{n}$ towards $\theta$
          when $n$ tends to infinity.

    \item Study the limit distribution of $\sqrt{n}\left(\hat{\theta}_{n}-\theta\right)$.

  \end{enumerate}
\end{Exercise}

\begin{solution}
  \begin{enumerate}
    \item
          Comme $Y$ prend ses valeurs dans $\{0,1\}$, $Y$ suit une loi de Bernoulli avec
          param\`etre $\proba{Y=1} = \proba{\theta>\xi} = \Phi(\theta)$

    \item
          Puisque $\frac1n\sum_{i=1}^n Y_i\to\esperance{}[Y_1]=\Phi(\theta)$ p.s et $\Phi^{-1}$ est
          une fonction continue, on a $\hat \theta_n=\Phi^{-1}(\frac1n\sum_{i=1}^n Y_i)\to
            \Phi^{-1}(\Phi(\theta))=\theta~p.s$. Donc $\hat\theta_n$ est consistant pour
          $\theta$.

    \item
          En vertu du TCL (car $\esperance{}[Y^2_1]<\infty$), on a $\sqrt
            n(\frac1n\sum_{i=1}^n Y_i-\Phi(\theta))\stackrel{\mathcal
              L}\longrightarrow\mathcal N(0,\var(Y_1))=\mathcal
            N(0,\Phi(\theta)(1-\Phi(\theta)))$. La fonction $\Phi^{-1}(\theta)$ est
          continument d\'erivable avec d\'eriv\'ee $(\Phi^{-1})'(\theta) =
            1/\varphi(\Phi^{-1}(\theta))$. On obtient par la delta-m\'ethode
          \begin{align*}
            \sqrt n(\hat\theta_n-\theta) =\sqrt n\left(\Phi^{-1}\left(\frac1n\sum_{i=1}^n Y_i\right)-\Phi^{-1}(\Phi(\theta))\right)
             & \stackrel{\mathcal L}\longrightarrow \mathcal N(0, {((\Phi^{-1})'(\Phi(\theta)))}^2   \Phi(\theta)(1-\Phi(\theta))) \\
             & =\mathcal N\left(0, \frac{  \Phi(\theta)(1-\Phi(\theta))}{ \varphi^2(\theta)}\right)
          \end{align*}

  \end{enumerate}

\end{solution}