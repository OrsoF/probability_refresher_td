\begin{center}
    \section*{Gamma distribution}
\end{center}

\begin{Exercise}
    (Gamma distribution). One says that $X$ has Gamma distribution with parameters $p>0$ et $\theta>0$, denoted by $\gamma(p, \theta)$, if its density is given by

$$
f(x)=\frac{\theta^{p}}{\Gamma(p)} \exp (-\theta x) x^{p-1} \mathbb{1}_{[0,+\infty[}(x) .
$$

The associated characteristic function is given by

$$
\Phi_{X}(t)=\frac{1}{(1-i t / \theta)^{p}}, \quad t \in \mathbb{R} .
$$

Here $\Gamma(\cdot)$ denotes the Gamma function defined as

$$
\forall \alpha>0, \quad \Gamma(\alpha)=\int_{0}^{\infty} x^{\alpha-1} \exp (-x) \mathrm{d} x, \quad \Gamma(\alpha+1)=\alpha \Gamma(\alpha), \quad \Gamma(1 / 2)=\sqrt{\pi} .
$$

\begin{enumerate}
  \item Compute $\mathbb{E}\left[X^{k}\right]$ for $k \geq 1$. Deduce that $\mathbb{E}[X]=p / \theta$ and $\operatorname{Var}(X)=p / \theta^{2}$.

  \item Let $a>0$. Show that $X / a \sim \gamma(p, a \theta)$.

  \item Let $X$ and $Y$ be two independent random variables with Gamma distribution $\gamma\left(p_{1}, \theta\right)$ and $\gamma\left(p_{2}, \theta\right)$, respectively. Show that $X+Y \sim \gamma\left(p_{1}+p_{2}, \theta\right)$.

  \item Let $Z$ have standard normal distribution $\mathcal{N}(0,1)$. What is the distribution of $Z^{2}$ ?

  \item Let $X_{1}, \ldots, X_{n}$ be $n$ i.i.d. random variables aléatoires with exponential distribution $\operatorname{Exp}(\theta)$. Determine the distribution of the sum $S_{n}=X_{1}+\ldots+X_{n}$. Compute $\mathbb{E}\left[S_{n}\right]$ and $\operatorname{Var}\left(S_{n}\right)$.

  \item Let $X_{1}, \ldots, X_{n}$ be $n$ i.i.d. random variables aléatoires with standard normal distribution $\mathcal{N}(0,1)$. Determine the distribution of the sum $S_{n}^{\prime}=X_{1}^{2}+\ldots+X_{n}^{2}$. Compute $\mathbb{E}\left[S_{n}^{\prime}\right]$ and $\operatorname{Var}\left(S_{n}^{\prime}\right)$.
\end{enumerate}

\end{Exercise}