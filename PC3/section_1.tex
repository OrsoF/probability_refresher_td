\begin{center}
  \section*{Gamma distribution}
\end{center}

%%% Exercise 1 %%%

\begin{Exercise}
  (Gamma distribution). One says that $X$ has Gamma distribution with parameters $p>0$ et $\theta>0$, denoted by $\gamma(p, \theta)$, if its density is given by
  \[
    f(x)=\frac{\theta^{p}}{\Gamma(p)} \exp (-\theta x) x^{p-1} \indicator{ [0,+\infty[ }(x)
  \]
  The associated characteristic function is given by
  \[
    \Phi_{X}(t)=\frac{1}{{(1-i t / \theta)}^{p}}, \quad t \in \mathbb{R} .
  \]
  Here $\Gamma(\cdot)$ denotes the Gamma function defined as
  \[
    \forall \alpha>0, \quad \Gamma(\alpha)=\int_{0}^{\infty} x^{\alpha-1} \exp (-x) \mathrm{d} x, \quad \Gamma(\alpha+1)=\alpha \Gamma(\alpha), \quad \Gamma(1 / 2)=\sqrt{\pi}
  \]

  \begin{enumerate}
    \item Compute $\mathbb{E}\left[X^{k}\right]$ for $k \geq 1$. Deduce that
          $\mathbb{E}[X]=p / \theta$ and $\operatorname{Var}(X)=p / \theta^{2}$.
    \item Let $a>0$. Show that $X / a \sim \gamma(p, a \theta)$.
    \item Let $X$ and $Y$ be two independent random variables with Gamma distribution
          $\gamma\left(p_{1}, \theta\right)$ and $\gamma\left(p_{2}, \theta\right)$,
          respectively. Show that $X+Y \sim \gamma\left(p_{1}+p_{2}, \theta\right)$.

    \item Let $Z$ have standard normal distribution $\mathcal{N}(0,1)$. What is the
          distribution of $Z^{2}$?

    \item Let $X_{1}, \ldots, X_{n}$ be $n$ i.i.d.\, random variables aléatoires with
          exponential distribution $\operatorname{Exp}(\theta)$. Determine the
          distribution of the sum $S_{n}=X_{1}+\cdots+X_{n}$. Compute
          $\mathbb{E}\left[S_{n}\right]$ and $\operatorname{Var}\left(S_{n}\right)$.

    \item Let $X_{1}, \ldots, X_{n}$ be $n$ i.i.d.\, random variables aléatoires with
          standard normal distribution $\mathcal{N}(0,1)$. Determine the distribution of
          the sum $S_{n}^{\prime}=X_{1}^{2}+\cdots+X_{n}^{2}$. Compute
          $\mathbb{E}\left[S_{n}^{\prime}\right]$ and
          $\operatorname{Var}\left(S_{n}^{\prime}\right)$.
  \end{enumerate}
\end{Exercise}

%%% Solution 1 %%%

\begin{solution}
  \begin{enumerate}
    \item We have:
          \begin{align*}
            \esperance{X^k} & = \frac{\theta^p}{\Gamma(p)} \int_0^\infty x^k e^{-\theta x} x^{p-1} dx \\
                            & = \frac{\theta^p}{\Gamma(p)} \frac{\Gamma(k+p)}{\theta^{k+p}}           \\
                            & = \frac{(k+p-1) \times \cdots \times p}{\theta^k}
          \end{align*}
          Therefore,
          \[\mathbb{E}[X]=p / \theta \]
          and
          \[\operatorname{Var}(X)=p / \theta^{2}\]
    \item For $a > 0$, we have:
          \begin{align*}
            \Phi_{X/a}(t) & = \esperance{e^{itX/a}}                   \\
                          & = \esperance{e^{i\frac{t}{a}X}}           \\
                          & = \frac{1}{{(1-i \frac{t}{a\theta})}^{p}}
          \end{align*}
          Therefore $X/a \sim \gamma(p, a\theta)$ identifying random variables from characteristic functions.
    \item We have
          \begin{align*}
            \Phi_{X+Y}(t) & = \Phi_{X}(t) \Phi_{Y}(t)                                                             \\
                          & = \frac{1}{{(1-i \frac{t}{a\theta})}^{p_1}} \frac{1}{{(1-i \frac{t}{a\theta})}^{p_2}} \\
                          & = \frac{1}{{(1-i \frac{t}{a\theta})}^{p_1 + p_2}}
          \end{align*}
          Which concludes.
    \item The distribution of $Z$ is exactly a Gamma distribution $\gamma(1/2, 1/2)$ (or a Chi-squared distribution $\chi^2_1$ with one degree of liberty):
          \[
            f_{Z^2} (x) = \frac{1}{\sqrt{2 \pi x} }e^{-x/2}
          \]
          Indeed, $\forall t \geq 0$,
          \begin{align*}
            \proba{Z^2 \leq x} & = \proba{Z \leq \sqrt{x}}                                                 \\
                               & = \int_0^{\sqrt{x}} \frac{1}{\sqrt{2\pi}} e^{-\frac{t^2}{2}} dt           \\
                               & = \int_0^{x} \frac{1}{\sqrt{2 \pi u} e^{-u/2}} \text{ with } u = \sqrt{t}
          \end{align*}
          which concludes.
    \item For $X$ following an exponential distribution,
          \[
            \Phi_X(t) = \frac{1}{1 - it/\theta}
          \]
          Therefore,
          \[
            \Phi_{S_n}(t) = {\left( \frac{1}{1 - it/\theta} \right)}^n
          \]
          And finally $S_n \sim \gamma(n, \theta)$ with same expected value and variance than in 1.
    \item Using question 3 and 4 for $n$ variables:
          $$ X_1^2 + \cdots + X_n^2 \sim \gamma(n/2, n/2) $$

  \end{enumerate}
\end{solution}