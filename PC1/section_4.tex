\begin{center}
  \section*{Expectation}
\end{center}

\begin{Exercise}
  Compute the mean, variance and cumulated distribution function of
  \vspace*{0.2cm}

  \begin{enumerate}
    \item the binomial distribution $\operatorname{Bin}(n, p)$ with $n \geq 1$ and $p>0$.

    \item the Poisson distribution $\operatorname{Poi}(\lambda)$ with $\lambda>0$.

    \item the uniform distribution $U[a, b]$ with $a<b$.

    \item the exponential distribution $\operatorname{Exp}(\lambda)$ with $\lambda>0$.

    \item the normal distribution $\mathcal{N}\left(\mu, \sigma^{2}\right)$ of probability density function
          \[
            f(x) : x \to \frac{1}{\sqrt{2\pi \sigma^2}} e^{- \frac{{(x-\mu)}^2}{2 \sigma^2}}
          \]
          with $\mu \in \mathbb{R}$ and $\sigma>0$.

  \end{enumerate}
\end{Exercise}

\begin{solution}
  \begin{enumerate}
    \item As any binomial random variable can be expressed as the sum of independant Bernoulli random variables, we have
          \begin{align*}
            \esperance{X} & = \esperance{\sum_{i=1}^{n} X_i} = \sum_{i=1}^{n} \esperance{X_i} = \sum_{i=1}^{n} p = np
          \end{align*}
          and
          \begin{align*}
            \var{X} & = \var{\sum_{i=1}^{n} X_i} \\
                    & = \sum_{i=1}^{n} \var{X_i} \\
                    & = \sum_{i=1}^{n} p(1-p)    \\
                    & = np(1-p)
          \end{align*}
    \item We have
          \begin{align*}
            \esperance{X} & = \frac{1}{\lambda} \int_{\mathbb{R}^+} x.e^{-\lambda.x} dx                                                                                                \\
                          & = \frac{1}{\lambda} {\left[ x \frac{e^{-\lambda.x}}{-\lambda} \right]}_0^\infty - \frac{1}{\lambda} \int_{\mathbb{R}^+} \frac{e^{-\lambda.x}}{-\lambda} dx \\
                          & = \frac{1}{\lambda}
          \end{align*}
          \begin{align*}
            \var{X} & = \frac{1}{\lambda} \int_{\mathbb{R}^+} x^2.e^{-\lambda.x} dx - \frac{1}{\lambda^2} \\
                    & = \frac{2}{\lambda^2} \int_{\mathbb{R}^+} x.e^{-\lambda.x} dx - \frac{1}{\lambda^2} \\
                    & = \frac{1}{\lambda^2}
          \end{align*}
    \item We have
          \begin{align*}
            \esperance{X} & = \int_{a}^{b} \frac{x}{b-a} dx                 \\
                          & = \frac{1}{b-a} \int_{a}^{b} x dx               \\
                          & = \frac{1}{b-a} (\frac{b^2}{2} - \frac{a^2}{2}) \\
                          & = \frac{a+b}{2}
          \end{align*}
  \end{enumerate}
\end{solution}

\begin{Exercise}
  \begin{enumerate}
    \item Show that if $X$ exponential distribution $\operatorname{Exp}(\lambda)$ with $\lambda>0$, then $\mathbb{E}\left[X^{n}\right]=\frac{n !}{\lambda^{n}}$;

    \item Show that if $X$ follows $\mathcal{N}(0,1)$ then $\mathbb{E}\left[X^{2 n}\right]=\prod_{k=1}^{n}(2 k-1)=\frac{(2 n) !}{2^{n} n !}$.

  \end{enumerate}
\end{Exercise}

\begin{solution}
  \begin{enumerate}
    \item Par int\'{e}gration par parties:
          \[
            \mathbb{E}\left(X^{n}\right) = \int_{0}^{+\infty}x^{n}\lambda e^{-\lambda x}dx = \int_{0}^{+\infty}x^{n-1} e^{-\lambda x}dx = \frac{n}{\lambda}\mathbb{E}\left(X^{n-1}\right).
          \]
          On en d\'{e}duit le r\'{e}sultat par r\'{e}curence imm\'{e}diate.
    \item Par int\'{e}gration par parties:
          \[
            \mathbb{E}\left(X^{2n}\right) = \int_{\mathbb{R}}x^{2n}\frac{e^{-\frac{x^2}{2}}}{\sqrt{2\pi}}dx = \int_{\mathbb{R}}\frac{x^{2n+2}}{2n + 1}\frac{e^{-\frac{x^2}{2}}}{\sqrt{2\pi}}dx = \frac{1}{2n+1}\mathbb{E}\left(X^{2(n+1)}\right).
          \]
          On en d\'{e}duit le r\'{e}sultat par r\'{e}curence imm\'{e}diate.
  \end{enumerate}
\end{solution}

\begin{Exercise}
  Let $X: \Omega \rightarrow[0 ;+\infty]$ (note that $+\infty$ is allowed) be a random variable such that $\mathbb{E}[X]<\infty$.

  \begin{enumerate}
    \item Prove that $X$ is finite almost surely (proceed by contradiction).

    \item Assume that $\mathbb{E}[X]=0$. Prove that $X=0$ almost surely. Hint: use that $X \geq \frac{1}{n} \mathbf{1}_{X \geq 1 / n}$.
  \end{enumerate}
\end{Exercise}

\begin{solution}
  \begin{enumerate}
    \item Let assume that $\proba{X = \infty} > 0$. Then:
          \[ \esperance{X} = \sum_{\omega \in \Omega} X(\omega) \proba{\omega} \]
          In the sum, there is $\omega \in \omega$ such that $X(\omega) = \infty$ and the sum cannot be finite.
    \item $X$ is positive and therefore, as a sum of positive terms:
          \[\esperance{X} = 0 \implies \sum_{\omega \in \Omega} X(\omega) \proba{\omega} = 0 \implies \forall \omega, X(\omega) = 0 \text{ or } P(\omega) = 0 \]
          Then ''$\forall \omega, X(\omega) = 0 \text{ or } P(\omega) = 0$'' can be red $\proba{X \geq 0} = 0$.
  \end{enumerate}
\end{solution}