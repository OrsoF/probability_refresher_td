\documentclass{article}

%%% Packages %%%

\usepackage[margin=1in]{geometry}
\usepackage{exercise}
\renewcommand{\ExerciseHeader}{%
  \par\noindent
  \textbf{\large \ExerciseName{} \, \ExerciseHeaderNB\ExerciseHeaderTitle\ExerciseHeaderOrigin}%
  \par\nopagebreak\medskip
}
\usepackage[utf8]{inputenc}
\usepackage[T1]{fontenc}
\usepackage{amsmath, amsfonts, amssymb}
\usepackage[version=4]{mhchem}
\usepackage{stmaryrd}
\usepackage{bbold} 

%%% Commands %%%

\newcommand{\proba}[1]{\mathbb{P}[#1]}
\newcommand{\var}[1]{\operatorname{Var}[#1]}
\newcommand{\esperance}[1]{\mathbb{E}[#1]}
\newcommand{\limitinftyn}{\xrightarrow[n \to{} \infty]{}}
\newcommand{\indicator}[1]{\mathbb{1}_{#1}}
\newcommand{\comb}[2]{\begin{pmatrix} #2 \\ #1 \end{pmatrix}}

%%% Solutions %%%

\newtheorem{solution}{Solution}
\newif\ifhidesolutions{}
% \hidesolutionstrue %decommenter pour cacher les SOLUTIONS

\ifhidesolutions{}
\usepackage{environ}
\NewEnviron{hide}{}
\let\solution\hide{}
\let\endsolution\endhide{}
\fi

%%% Title %%%

\title{PC 1 – Sets, Measures and Random Variables}
\author{}
\date{}

%%% Document %%%

\begin{document}

\begin{flushleft}
  \textbf{Probability Refresher} \hfill \textbf{September 2023} \\
  \textbf{Master X-HEC} \hfill \\
  É\textbf{cole Polytechnique} \hfill
\end{flushleft}

{\let\newpage\relax\maketitle}
\vspace{-1.3cm}
\hrule

\vspace{0.5cm}
\begin{center}
  \section*{Convergence}
\end{center}

\begin{Exercise}
  Let $\left\{X_{i}\right\}_{i \geq 0}$ be a sequence of i.i.d. Bernoulli variables with parameter $\theta$.

  \begin{enumerate}
    \item Show that $\sqrt{n}\left(\bar{X}_{n}-\theta\right) \stackrel{d}{\longrightarrow} \mathcal{N}(0, \theta(1-\theta))$, where $\bar{X}_{n}=n^{-1} \sum_{i=1}^{n} X_{i}$.

    \item Show that $\bar{X}_{n}\left(1-\bar{X}_{n}\right) \stackrel{P}{\longrightarrow} \theta(1-\theta)$.

    \item Show that $\sqrt{n}\left(\bar{X}_{n}-\theta\right)^{2} \stackrel{P}{\longrightarrow} 0$.

    \item Determine the limit distribution of $\sqrt{n}\left(\bar{X}_{n}\left(1-\bar{X}_{n}\right)-\theta(1-\theta)\right)$.

  \end{enumerate}
\end{Exercise}

\begin{Exercise}
  Let $\left(X_{n}\right)_{n \geq 1}$ be a sequence of i.i.d. square-integrable random variables with mean $m$ and variance $\sigma^{2}>0$. Denote $\bar{X}_{n}=\frac{1}{n} \sum_{i=1}^{n} X_{i}$ and $\hat{\sigma}_{n}^{2}=\frac{1}{n} \sum_{i=1}^{n}\left(X_{i}-\bar{X}_{n}\right)^{2}$.

  \begin{enumerate}
    \item Show that $\hat{\sigma}_{n}^{2}$ converges in probability to $\sigma^{2}$ as $n \rightarrow \infty$.

    \item Determine the limit distribution of $\sqrt{n}\left(\bar{X}_{n}-m\right) / \hat{\sigma}_{n}$.

  \end{enumerate}
\end{Exercise}


\begin{Exercise}
  (Poisson model). Let $\left(X_{1}, \ldots, X_{n}\right)$ be an i.i.d. sample from the Poisson distribution with unknown parameter $\lambda>0$. Denote $\bar{X}_{n}=\frac{1}{n} \sum_{i=1}^{n} X_{i}$.

  \begin{enumerate}
    \item Show that $\bar{X}_{n}$ is an unbiased estimator of $\lambda$, that is $\mathbb{E}\left[\bar{X}_{n}\right]=\lambda$.

    \item Show that $\bar{X}_{n}$ converges in probability to $\lambda$ when $n$ tends to infinity.

    \item Determine the limit distribution of $\sqrt{n}\left(\bar{X}_{n}-\lambda\right) / \sqrt{\bar{X}_{n}}$.

    \item Find an appropriate function $g$ such that $\sqrt{n}\left(g\left(\bar{X}_{n}\right)-g(\lambda)\right) \stackrel{d}{\longrightarrow} \mathcal{N}(0,1)$.

  \end{enumerate}
\end{Exercise}


\begin{Exercise}
  Define the random variable

  $$
    Y=\mathbb{1}\{\theta>X\}
  $$

  where $\theta \in \mathbb{R}$ and $X$ is a random variable with standard normal distribution $\mathcal{N}(0,1)$. We observe a sample $Y_{1}, \ldots, Y_{n}$ of i.i.d. realizations of $Y$ and suppose that parameter $\theta$ is unknown. Denote by $\Phi$ the cumulative distribution function of the standard normal distribution $\mathcal{N}(0,1)$. An estimator $\hat{\theta}_{n}$ of $\theta$ is given by

  $$
    \hat{\theta}_{n}=\Phi^{-1}\left(\bar{Y}_{n}\right)
  $$

  where $\bar{Y}_{n}=\frac{1}{n} \sum_{i=1}^{n} Y_{i}$

  \begin{enumerate}
    \item Determine the distribution of $Y$.

    \item Study the convergence in probability of $\hat{\theta}_{n}$ towards $\theta$ when $n$ tends to infinity.

    \item Study the limit distribution of $\sqrt{n}\left(\hat{\theta}_{n}-\theta\right)$.

  \end{enumerate}
\end{Exercise}
\begin{center}
  \section*{Conditionning}
\end{center}

\begin{Exercise}
  Let $(X, Y)$ be a couple of random variables admitting a density on $\mathbb{R}^{2}$ such that

  (i) $X$ has Gamma distribution $\gamma(2, \lambda)$ (with density $f_{X}(x)=\lambda^{2} x \mathrm{e}^{-\lambda x} \mathbb{1}_{\{x \geq 0\}}$ ),

  (ii) the conditional distribution of $Y$ given $X$ is the uniform distribution on the segment $[0, X]$ (or, to put it differently, the conditional density of $Y$ given $X=x$ is $f_{Y \mid X=x}(y)=$ $\left.\frac{1}{x} \mathbb{1}_{\{0<y<x\}}\right)$.

  \begin{enumerate}
    \item Determine the density of $(X, Y)$ and the distribution of $Y$.

    \item Compute the conditional density of $X$ given $Y$.

    \item Evaluate the following quantities:

  \end{enumerate}

  (a) $\mathbb{E}[X Y]$ (one may use that $\mathbb{E}\left[X^{2}\right]=\frac{6}{\lambda^{2}}$ ),

  (b) $\mathbb{E}[Y \mid X]$,

  (c) $\mathbb{E}[X \mid Y]$,

  (d) $\mathbb{E}[X+X Y \mid Y]$,

  (e) $\mathbb{E}[\mathbb{E}[Y \mid X]]$
\end{Exercise}

\begin{Exercise}
  We consider an electronic device. Denote by $T$ the life time of the component, which is the amount of time in years such that it works properly until it breaks down. We assume that $T$ is exponentially distributed with parameter $1 / 2$.

  \begin{enumerate}
    \item What is the probability that the device breaks down during the first year?

    \item We know that the electronic device has been used during two years without any problems. What is the probability that it breaks down during the third year?

  \end{enumerate}
\end{Exercise}

\begin{Exercise}
  We set

  $$
    f(x, y)=\frac{\alpha y^{2}}{x} \mathbf{1}_{0<y<x<1} .
  $$

  \begin{enumerate}
    \item Find $\alpha$ such that $f$ is a probability density on $\mathbb{R}^{2}$ of some pair of random variables $(X, Y)$ and compute the marginal laws of $X$ and $Y$.

    \item Prove that the conditional density of $Y$ given $X=x$ is given by

  \end{enumerate}

  $$
    f_{Y \mid X=x}(y)=\frac{3 y^{2}}{x^{3}}, 0<y<x<1
  $$

  and deduce that $\mathbb{E}[Y \mid X]=\frac{3}{4} X$.
\end{Exercise}
\begin{center}
  \section*{Random variables}
\end{center}

\begin{Exercise}
  Find two random variables $X$ and $Y$ on a probability space $(\Omega, \mathbb{P})$ (to be specified) having the same distribution, but that are not equal.
\end{Exercise}

\begin{Exercise}
  In an oil region, the probability that one drilling leads to an oil slick is 0.1 .
  \begin{enumerate}
    \item Justify that one drilling can be modeled using a Bernoulli distribution.
    \item We made 10 oil drillings. Let $X$ be the number of drillings that led to an oil slick.
          \begin{enumerate}
            \item Under which assumptions $X$ can be modeled using a binomial distribution? Precise the parameters.
            \item Assume that $X$ follows a binomial distribution. Compute
                  \begin{enumerate}
                    \item the probability that exactly two drillings lead to oil slicks.
                    \item the probability that at least one drilling leads to an oil slick.
                  \end{enumerate}
          \end{enumerate}
  \end{enumerate}
\end{Exercise}

\begin{Exercise}
  Let $\lambda>0$ be fixed. Let $X_{n}, n \geq 1$ be random variables with binomial distribution with parameters $n$ and $\lambda / n$, and $Y$ be a random variable with Poisson distribution with parameter $\lambda$. Show that, for any $k \in \mathbb{N}$,

  $$
    \lim _{n \rightarrow+\infty} \mathbb{P}\left(X_{n}=k\right)=\mathbb{P}(Y=k) .
  $$

  Hint: Use Striling's approximation: $n ! \approx \sqrt{2 \pi n}\left(\frac{n}{\mathrm{e}}\right)^{n}$.

  We will later see that this result means that $X_{n}$ converges in distribution to $Y$, or, to put it differently, that the binomial distribution with parameters $n$ and $\lambda / n$ converges to the Poisson distribution with parameter $\lambda$.
\end{Exercise}

\begin{center}
  \section*{Expectation}
\end{center}

\begin{Exercise}
  Compute the mean, variance and cumulated distribution function of
  \vspace*{0.2cm}

  \begin{enumerate}
    \item the binomial distribution $\operatorname{Bin}(n, p)$ with $n \geq 1$ and $p>0$.

    \item the Poisson distribution $\operatorname{Poi}(\lambda)$ with $\lambda>0$.

    \item the uniform distribution $U[a, b]$ with $a<b$.

    \item the exponential distribution $\operatorname{Exp}(\lambda)$ with $\lambda>0$.

    \item the normal distribution $\mathcal{N}\left(\mu, \sigma^{2}\right)$ of probability density function
          $$ f(x) : x \to \frac{1}{\sqrt{2\pi \sigma^2}} e^{- \frac{(x-\mu)^2}{2 \sigma^2}}$$
          with $\mu \in \mathbb{R}$ and $\sigma>0$.

  \end{enumerate}
\end{Exercise}

\begin{Exercise}
  \begin{enumerate}
    \item Show that if $X$ exponential distribution $\operatorname{Exp}(\lambda)$ with $\lambda>0$, then $\mathbb{E}\left[X^{n}\right]=\frac{n !}{\lambda^{n}}$;

    \item Show that if $X$ follows $\mathcal{N}(0,1)$ then $\mathbb{E}\left[X^{2 n}\right]=\prod_{k=1}^{n}(2 k-1)=\frac{(2 n) !}{2^{n} n !}$.

  \end{enumerate}
\end{Exercise}

\begin{solution}
  \begin{enumerate}
    \item Par int\'{e}gration par parties :
          \[
            \mathbb{E}\left(X^{n}\right) = \int_{0}^{+\infty}x^{n}\lambda e^{-\lambda x}dx = \int_{0}^{+\infty}x^{n-1} e^{-\lambda x}dx = \frac{n}{\lambda}\mathbb{E}\left(X^{n-1}\right).
          \]
          On en d\'{e}duit le r\'{e}sultat par r\'{e}curence imm\'{e}diate.
    \item Par int\'{e}gration par parties:
          \[
            \mathbb{E}\left(X^{2n}\right) = \int_{\mathbb{R}}x^{2n}\frac{e^{-\frac{x^2}{2}}}{\sqrt{2\pi}}dx = \int_{\mathbb{R}}\frac{x^{2n+2}}{2n + 1}\frac{e^{-\frac{x^2}{2}}}{\sqrt{2\pi}}dx = \frac{1}{2n+1}\mathbb{E}\left(X^{2(n+1)}\right).
          \]
          On en d\'{e}duit le r\'{e}sultat par r\'{e}curence imm\'{e}diate.
  \end{enumerate}
\end{solution}

\begin{Exercise}
  ${ }^{*}$ Let $X: \Omega \rightarrow[0 ;+\infty]$ (note that $+\infty$ is allowed) be a random variable such that $\mathbb{E}[X]<\infty$.

  \begin{enumerate}
    \item Prove that $X$ is finite almost surely (proceed by contradiction).

    \item Assume that $\mathbb{E}[X]=0$. Prove that $X=0$ almost surely. Hint: use that $X \geq \frac{1}{n} \mathbf{1}_{X \geq 1 / n}$.
  \end{enumerate}
\end{Exercise}

\begin{center}
    \section*{Variance Inequalities}
\end{center}

\begin{Exercise}
    Let $X$ be a random variable such that $\mathbb{E}\left[X^2\right]<+\infty$. Prove that :
    \begin{enumerate}
        \item $0 \leq \operatorname{Var}(X)<\infty$
        \item $\operatorname{Var}(X)=\mathbb{E}\left[X^2\right]-(\mathbb{E}[X])^2$.
        \item $\operatorname{Var}(X)=0 \Longleftrightarrow \mathbb{P}(X=c)=1$ for some constant $c$.
        \item For any constants $a, b, \operatorname{Var}(a X+b)=\operatorname{Var}(a X)=a^2 \operatorname{Var}(X)$.
    \end{enumerate}
\end{Exercise}

\begin{Exercise}
    Let $X$ be non-negative ($X \geq 0$ a.s.) and $a>0$ be a constant.
    \begin{enumerate}
        \item Justify that
              $$\forall \omega \in \Omega, \quad a \mathbf{1}_{\{Z(\omega) \geqslant a\}} \leqslant Z(\omega) \mathbf{1}_{\{Z(\omega) \geqslant a\}} \leqslant Z(\omega)$$
        \item Prove the Markov's inequality
              $$
                  \mathbb{P}(X \geq a) \leq \frac{\mathbb{E}[X]}{a}
              $$
    \end{enumerate}
\end{Exercise}

\begin{Exercise}
    Assume that $\mathbb{E}\left[X^2\right]<+\infty$. Applying Markov's inequality to $(X-\mathbb{E}[X])^2$ prove that, for any constant $a>0$,
    $$
        \mathbb{P}(|X-\mathbb{E}[X]| \geq a) \leq \frac{\operatorname{Var}(X)}{a^2}
    $$
\end{Exercise}

\end{document}