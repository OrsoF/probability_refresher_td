\begin{center}
  \section*{Random variables}
\end{center}

\begin{Exercise}
  Find two random variables $X$ and $Y$ on a probability space $(\Omega, \mathbb{P})$ (to be specified) having the same distribution, but that are not equal.
\end{Exercise}

\begin{solution}
  Let $X$ a random variable on $[1/2, 2]$ with p.d.f. $f : x \to \frac{2}{3x}$ and $Y = 1/X$.
  \begin{enumerate}
    \item $X$ and $1/X$ have same p.d.f.
    \item $X$ and $1/X$ are not the same random variable. Indeed:
          $$\frac{X}{1/X} = X^2$$
          and $X^2$ is not equal to $1$ almost surely.
  \end{enumerate}
\end{solution}

\begin{Exercise}
  In an oil region, the probability that one drilling leads to an oil slick is 0.1 .
  \begin{enumerate}
    \item Justify that one drilling can be modeled using a Bernoulli distribution.
    \item We made 10 oil drillings. Let $X$ be the number of drillings that led to an oil slick.
          \begin{enumerate}
            \item Under which assumptions $X$ can be modeled using a binomial distribution? Precise the parameters.
            \item Assume that $X$ follows a binomial distribution. Compute
                  \begin{enumerate}
                    \item the probability that exactly two drillings lead to oil slicks.
                    \item the probability that at least one drilling leads to an oil slick.
                  \end{enumerate}
          \end{enumerate}
  \end{enumerate}
\end{Exercise}

\begin{solution}
  \begin{enumerate}
    \item Bernoulli is a Success/Failure model with a given probability of succes.
    \item Using definition of a binomial law :
          \begin{enumerate}
            \item \begin{align*}
                    \proba{X = 2} & = \binom{10}{2} .1^2 \times .9^8 \\
                                  & \approx 0.194
                  \end{align*}
            \item \begin{align*}
                    \proba{X \geq 1} & = 1 - \proba{X = 0}                \\
                                     & = 1 - \binom{10}{0} 0.1^0 0.9^{10} \\
                                     & \approx 0.651
                  \end{align*}
          \end{enumerate}
  \end{enumerate}
\end{solution}

\begin{Exercise}
  Let $\lambda>0$ be fixed. Let $X_{n}, n \geq 1$ be random variables with binomial distribution with parameters $n$ and $\lambda / n$, and $Y$ be a random variable with Poisson distribution with parameter $\lambda$. Show that, for any $k \in \mathbb{N}$,

  \[
    \lim _{n \rightarrow+\infty} \mathbb{P}\left(X_{n}=k\right)=\mathbb{P}(Y=k) .
  \]

  Hint: Use Striling's approximation: $n ! \approx \sqrt{2 \pi n}{\left(\frac{n}{\mathrm{e}}\right)}^{n}$.

  We will later see that this result means that $X_{n}$ converges in distribution to $Y$, or, to put it differently, that the binomial distribution with parameters $n$ and $\lambda / n$ converges to the Poisson distribution with parameter $\lambda$.
\end{Exercise}

\begin{solution}
  We have:
  \begin{align*}
    \proba{X_n = k} & = \binom{n}{k} \left( \frac{\lambda}{n} \right)^k \left(  1- \frac{\lambda}{n} \right)^{n-k}        \\
                    & = \frac{n!}{k!(n-k)!} \left( \frac{\lambda}{n} \right)^k \left(  1- \frac{\lambda}{n} \right)^{n-k} \\
                    & = \frac{\lambda^k}{k!} \frac{n!}{(n-k)! n^k} \left(  1- \frac{\lambda}{n} \right)^{n-k}
  \end{align*}
  And:
  \[
    \left(  1- \frac{\lambda}{n} \right)^{n-k} = {\left(  1- \frac{\lambda}{n} \right)}^{k} \left(  1- \frac{\lambda}{n} \right)^{n}
  \]
  with
  \[
    \left(  1- \frac{\lambda}{n} \right)^{k} \limitinftyn 0
  \]
  and
  \begin{align*}
    \left(  1- \frac{\lambda}{n} \right)^n & = e^{\log\left(  1- \frac{\lambda}{n} \right)^n}  \\
                                           & = e^{n.\log\left(  1- \frac{\lambda}{n} \right)}  \\
                                           & \thicksim e^{n\left( - \frac{\lambda}{n} \right)} \\
                                           & = e^{-\lambda}
  \end{align*}
  Also,
  \[ \frac{n!}{(n-k)!}= n.(n-1).(n-2). \cdot . (n-k+1) \thicksim n^k \]
  Finally,
  \[
    \binom{n}{k} \left( \frac{\lambda}{n} \right)^k \left(  1- \frac{\lambda}{n} \right)^{n-k} \thicksim \frac{\lambda^k}{k!} 1. e^{-\lambda}
  \]
  This conclude the exercise.
\end{solution}