\begin{center}
  \section*{Independence}
\end{center}

\begin{Exercise}[origin=Independent events]
  Let $\Omega=\left\{\omega_{1}, \omega_{2}, \omega_{3}, \omega_{4}\right\}$ equipped with the uniform probability distribution $\mathbb{P}$. Define the events $A=\left\{\omega_{1}, \omega_{2}\right\}$, $B=\left\{\omega_{1}, \omega_{3}\right\}$ and $C=\left\{\omega_{2}, \omega_{3}\right\}$.
  Show that $A, B$ and $C$ are pairwise independent. Compare $\mathbb{P}(A \cap B \cap C)$ and $\mathbb{P}(A) \mathbb{P}(B) \mathbb{P}(C)$.
\end{Exercise}

\vspace{0.5cm}

\begin{solution}
  We have:
  \[
    A \cap B = \{\omega_1, \omega_2\} \cap \{\omega_1, \omega_3\} =  \{\omega_1\}
  \]
  Therefore
  \[
    \proba{A \cap B} = \proba{\omega_1} = \frac{1}{4}
  \]
  And
  \[
    \proba{A}. \proba{B} = \frac{1}{2} . \frac{1}{2} = \frac{1}{4}
  \]
  Therefore:
  \[
    \proba{A \cap B} = \proba{A}.\proba{B}
  \]
  And finally $A$ and $B$ are independent. For similar reasons, $B$ and $C$ are independent and so $A$ and $C$.

  As $A \cap B \cap C=\emptyset$, we have  $\proba(A \cap B \cap C) =0$. However, $\proba(A) \proba(B) \proba(C)=1/8$. This implies that $A$, $B$ et $C$ are not mutually independent.
\end{solution}

\begin{Exercise}
  Let $A_1, \ldots , A_n$ be $n$ events from a probability space $(\Omega, \mathbb{P})$.
  Suppose that they are mutually independent. Find an explicit expression for $\mathbb{P}(A_1 \cup \cdots \cup A_n)$ depending on the $\mathbb{P}(A_i)$.
\end{Exercise}

\begin{solution}
  \begin{align*}
    \proba{A_1 \cup \cdots \cup A_n} & = 1- \proba{A_1^c \cap \cdots \cap A_n^c}             \\
                                     & = 1- \prod_{i = 1}^{n} \proba{A_i^c}                  \\
                                     & = 1- \prod_{i = 1}^{n} \left( 1 - \proba{A_i} \right)
  \end{align*}
\end{solution}

\begin{Exercise}
  Let $(\Omega, \mathcal{F}, \mathbb{P})$ a probability space. Let ${\left(A_n\right)}_{n \geq 0}$ a series of independent events. We note $A=\lim \sup _n A_n$. Let assume that $\sum_n \mathbb{P}\left(A_n\right)=+\infty$ and we want to prove that $\mathbb{P}(A)=1$.
  \begin{enumerate}
    \item Preliminary. Justify that for all $x>-1, \ln (1+x) \leq x$.
    \item Let $n \leq N$. We note $E_{n, N}=\bigcap_{k=n}^N A_k^c$ and $E_n=\bigcap_{k \geq n} A_k^c$.
          \begin{enumerate}
            \item Prove that ($n$ fixed), $\lim _{N \rightarrow+\infty} \ln \left(\mathbb{P}\left(E_{n, N}\right)\right)=-\infty$.
            \item Deduce that $\mathbb{P}\left(E_n\right)=0$.
            \item Deduce that $\mathbb{P}(A)=1$.
          \end{enumerate}
  \end{enumerate}
\end{Exercise}

\begin{solution}
  \begin{enumerate}
    \item La fonction ln est concave. Sa courbe représentative est en-dessous de sa tangente au point d'abscisse 1. L'inégalité demandée est juste la traduction analytique de cette propriété géométrique.
    \item \begin{enumerate}
            \item Les événements $A_k$ étant indépendants, il en est de même des événements $\overline{A_k}$, et donc
                  \[
                    P\left(E_{n, N}\right)=\prod_{k=n}^N P\left(\overline{A_k}\right)=\prod_{k=n}^N\left(1-P\left(A_k\right)\right) .
                  \]
                  En utilisant l'inégalité précédente, on a
                  \[
                    \ln \left(P\left(E_{n, N}\right)\right) \leq-\sum_{k=n}^N P\left(A_k\right) .
                  \]
                  Puisque $\sum_{k \geq n} P\left(A_k\right)=+\infty$, on en déduit le résultat.
            \item Par composition par la fonction exponentielle, $\left(P\left(E_{n, N}\right)\right)$ tend vers 0 lorsque $N$ tend vers l'infini (et $n$ reste fixé). Mais, la suite ${\left(E_{n, N}\right)}_N$ est décroissante et
                  \[
                    E_n=\bigcap_{N \geq n} E_{n, N}
                  \]
                  Ainsi,
                  \[
                    P\left(E_n\right)=\lim _N P\left(E_{n, N}\right)=0
                  \]
            \item $A$ s'écrit $A=\bigcap_n \overline{E_n}$. La suite $\left(\overline{E_n}\right)$ est décroissante et $P\left(\overline{E_n}\right)=1$. Ainsi, on trouve que
                  \[
                    P(A)=\lim _n P\left(\overline{E_n}\right)=1
                  \]
          \end{enumerate}
  \end{enumerate}
\end{solution}